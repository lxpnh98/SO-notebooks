\documentclass[12pt]{report}

\usepackage[a4paper]{geometry}
\usepackage[utf8]{inputenc}
\usepackage[portuguese]{babel}

\newcommand\tab[1][0.5cm]{\hspace*{#1}}

\title{Projeto de Sistemas Operativos (MiEI) \\ 2017/2018}
\author{Alexandre Mendonça Pinho (a82441) \and Joel Filipe Esteves Gama (a82202) \and Tiago Martins Pinheiro (a82491)}
\date{\today}

\begin{document}
\maketitle

\tableofcontents

\chapter{Introdução}
\label{sec:introducao}

\tab Neste projeto é-nos proposto construir um sistema para processamento de notebooks, que misturam fragmentos de código, resultados da execução, e documentação. Estes notebooks, tratam-se de ficheiros de texto, que depois de processados, são modificados de modo a incluir resultados da execução de código ou comandos nele embebidos.

\chapter{Descrição do projeto}
\label{sec:descricao}


\chapter{Conclusão}
\label{sec:conclusao}

\tab Foi-nos prosposto, como projeto de avaliação, conceber um sistema que fosse capaz de processar \textit{notebooks}, através da sua leitura e  interpretação. A implementação foi feita na linguagem de programação \textit{C}, utilizando os conceitos aprendidos durante as aulas práticas ao longo do semestre. 

Para os métodos de comunicação entre processos, escolhemos pipes anónimos para comunicar entre os processos e os distribuidores, e \textix{fifos} para comunicar entre os diferentes processos através dos distribuidores. Assim, a comunicação entre processos é tratada de forma simples pelos distribuidores.

\end{document}
