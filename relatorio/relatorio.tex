\documentclass[12pt]{report}

\usepackage[a4paper]{geometry}
\usepackage[utf8]{inputenc}
\usepackage[portuguese]{babel}

\newcommand\tab[1][0.5cm]{\hspace*{#1}}

\title{Projeto de Sistemas Operativos (MiEI) \\ 2017/2018}
\author{Alexandre Mendonça Pinho (a82441) \and Joel Filipe Esteves Gama (a82202) \and Tiago Martins Pinheiro (a82491)}
\date{\today}

\begin{document}
\maketitle

\tableofcontents

\chapter{Introdução}
\label{sec:introducao}

\tab Neste projeto é-nos proposto construir um sistema para processamento de notebooks, que misturam fragmentos de código, resultados da execução, e documentação. Estes notebooks, tratam-se de ficheiros de texto, que depois de processados, são modificados de modo a incluir resultados da execução de código ou comandos nele embebidos.

\chapter{Descrição do projeto}
\label{sec:descricao}

\tab Para resolver o problema que nos foi apresentado começamos por fazer o parsing do ficheiro. O parsing excluí o output dos comandos e constroi o grafo de execução. Num grafo de execução os nodos correspondem a comandos e as arestas são os redicionamentos dos comandos. Se um nodo A do grafo tem uma aresta vinda de um nodo B significa que o output de B será redirecionado como o input de A, se o nodo for isolado ou não tiver arestas vindas de outros nodos significa que o comando recebe o input do standard input. Já o que é para escrever no ficheiro final é mantido em memória central com recurso a uma lista ligada de strings.

Depois do parsing passamos para a execução dos comandos. Cada comando cria um pipe que é usado para distribuir o output do comando. A distribuição é feita através do programa distribuidor que lê do pipe e escreve nos fifos onde o output é necessário.

No final da execução de cada comando os outputs são colocados na lista ligada de strings, (cada comando tem um índice que indica qual a posição onde deve ser adicionado na lista ligada) que está em memória central, para serem escritos em ficheiro.

\chapter{Conclusão}
\label{sec:conclusao}

\end{document}
